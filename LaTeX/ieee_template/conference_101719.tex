\documentclass[conference]{IEEEtran}
\IEEEoverridecommandlockouts
% The preceding line is only needed to identify funding in the first footnote. If that is unneeded, please comment it out.
\usepackage{cite}
\usepackage{amsmath,amssymb,amsfonts}
\usepackage{algorithmic}
\usepackage{graphicx}
\usepackage{textcomp}
\usepackage[spanish]{babel}
\usepackage{xcolor}
\renewcommand{\baselinestretch}{1.5} %Modificar el interlineado

\def\BibTeX{{\rm B\kern-.05em{\sc i\kern-.025em b}\kern-.08em
    T\kern-.1667em\lower.7ex\hbox{E}\kern-.125emX}}
\begin{document}

\title{Tipos de Memorias de un Computador}

\author{\IEEEauthorblockN{Emmanuel Sol\'is}
\IEEEauthorblockA{\textit{Escuela de Ciencias de la Computaci\'on e Inform\'atica} \\
\textit{Universidad de Costa Rica}\\
San Jos\'e, Costa Rica\\
emmanuel.solisp@icloud.com}
}

\maketitle

\begin{abstract} %Resumen.
Escribir resumen (abstract).

\end{abstract}

\section{Introducci\'on}
Pequena instrouduccion respecto a los temas que desea incluir. Por ejemplo:
\begin{itemize}
    \item \textbf{Memoria de acceso aleatorio (RAM).}
    \item \textbf{Memoria solo de lectura (ROM).}
    \item \textbf{Memoria de acceso aleatorio de video (VRAM).}
    \item \textbf{Disco de estado solido (SSD).}
    \item \textbf{Matriz de puerta programable en campo (FPGA).}
    \item \textbf{Memorias especiales}
\end{itemize}
Terminar la pequena explicacion.

\section{Desarrollo}

\subsection{Planteamiento de problemas} %Problemas que se plantean solucionar.
El problema planteado en esta investigaci\'on es la falta de conocimiento respecto a las memorias
del computador, es por ello que a trav\'es de esta investigaci\'on se espera dar soluci\'on a
dicho problema, al hacer una investigaci\'on relevante que pueda dar el conocimiento necesario
respecto a los tipos de memoria de las computadoras.

\subsection{Soluci\'on} %Seccion de las soluciones a los problemas.
Dado que debe existir un orden en pro de tener un mejor aprendizaje la explicaci\'on estar\a'
dividida en la historia de cada una, sus caracter\'isticas arquitect\'onicas y explicar cada
una de ellas; esto adem\'as de explicar respecto a las memorias del futuro.

\subsubsection{Caracter\'isticas Arquitect\'onicas} %Primera Parte
\begin{itemize}
    \item \textbf{Memoria de acceso aleatorio (RAM):} La memoria RAM es un tipo de
    \textbf{memoria interna} que esta hecha de forma que sus datos son almacenados en 
    celdas con condensadores, en sus inicios estos eran de grandes tama\~nos, estos almacenan
    sus valores al estar cargados o no cargados, dichos estados son representados por 0's y 1's,
    es decir usando \'algebra booleana. Su composici\'on esta dividida en dos formas,
    \textit{RAM Din\'amica (DRAM)}, o la \textit{RAM Est\'atica (SRAM)}.
    La \textit{RAM Din\'amica (DRAM)} su nombre din\'amico se debe a su tendencia a que se den
    fugas de memoria, se caracteriza por ser de un costo de producci\'on bajo pero a menor
    rendimiento; mientras que la \textit{RAM Est\'atica (SRAM)} en contraste es de un costo mayor
    pero con mayores velocidides. Su composicion arquitect\'onica se muestra en la Figura.
    %\begin{figure}
    %    \centering
    %    \includegraphics[width=0.45\textwidth]{DRAM-SRAM.png}
    %    \caption{Estructuras de celdas de memoria.}
    %    \label{fig:RAM}
    %\end{figure}
    
    \item \textbf{Memoria solo de lectura (ROM):} Es una \textbf{memoria interna}, y seg\'un lo
    indica su nombre es solo de lectura. Normalmente las memorias tienen la capaciddad de lectura
    y escritura, es decir, seg\'un se requiere se puede guardar informaci\'on o bien leer dicha
    informaci\'on; en el caso de esta memoria solo se puede leer, los datos que en ella vienen
    almacenados han sido all\'i almacenados desde su fabricaci\'on, estos son utilizados para el
    arranque de las computadoras y su funcionamiento principal. Ahora bien, el hecho de que solo
    posea la capacidad de almacenar datos plantea la problem\'atica de que no existe ni m\'argen
    de error, es decir que si existe un solo \textit{bit} malo se debe desechar toda la memoria,
    y que los costos de producci\'on encarecen; pero a la vez permite la ventaje de conservar los
    datos que son necesarios para el funcionamiento de la computadora sin importar el manejo que
    le de el usuario a la m\'aquina, esto es muy relevante para asegurar su funcionamiento.
    
    \item \textbf{Memoria de acceso aleatorio de video (VRAM):} este es un tipo de
    \textbf{memoria interna}, es una memoria RAM pero para las \textit{tarjetas gr\'aficas}.
    Se comporta de forma especial dado que tiene la capacidad de, primero, resolver la distribuci\'on
    de recursos de la computadora para gr\'aficos de gran tama\~no y por ende de mucho consumo de
    recursos; y segundo por tener la capacidad de escritura y lectura al mismo tiempo, esto porque
    mientras que es leida por los monitores para mostrar al usuario los gr\'aficos que se han
    procesado, va procesando los pr\'oximos gr\'aficos de procesamiento.
    
    \item \textbf{Disco de estado solido (SSD):} este tipo de \textbf{memoria externa};
    podr\'iamos hablar de este como el sucesor del \textit{disco duro convencional (HDD)}.
    El disco duro convencional funcionaba con un disco magn\'etico esto en sus inicios fue
    una soluci\'on correcto pero la computaci\'on ha avanzado y hoy dia resulta ineficiente
    por sus bajos niveles de lectura, es por ello que existe el nuevo \textit{disco duro de
    estado solido} que utiliza chips para almacenar la informaci\'on. Este nuevo sistema de
    almacenamiento trae como ventajas frente al HDD que tienen velocidades de escrtitura y
    lectura mayores, tiene mayor vida \'util y tiene un menor consumo de energ\'ia. Podemos
    ver unos ejemplos de los discos SSD en la Figura XX.
    
    \item \textbf{Matriz de puerta programable en campo (FPGA):} estas son un tipo especial de
    memorias RAM que permiten ser reprogramadas para satisfacer las funciones requeridas en el
    momento, esta es su principal ventaja dado que su uso implica mayores costos de producci\'on y
    una menor eficiencia energ\'etica, es por ello que son solo usados cuando existe la necesidad
    de que los dispositivos puedan ser reprogramables. La arquitectura de este tipo de memoria se
    muestra en la Figura.
    
    \item \textbf{Memorias del futuro:} estas son un tipo de memoria flash; las memorias flash son
    un tipo de memoria que se comporta de forma interna y externa al mismo tiempo posee bajo costo
    y una funcionalidad eficiente para sus costos. Para entender que se desea con este tipo de
    memoria hay que ver la historia de la computaci\'on y comprender que la computaci\'on ha tenido
    un avance exponencial, en muy poco tiempo ha avanzado mucho, reduciendo los tama\~nos de sus
    componentes y aumentando sus capacidades; es por ello que hoy en d\'ia se propone un nuevo
    material para la fabricaci\'on de memorias, el cu\'al es el \textbf{grafeno}, un elemento que
    se espera pueda ser usado para la fabricaci\'on y aumentar dichas velocidades. 
\end{itemize}

\section{Conclusi\'on}
Podemos concluir que el conocimiento de los tipos de memoria de un sistema de computaci\'on nos
permite, tanto como usuario como profesionales, comprender el funcionamiento de las computadoras
y al comprender esto podemos mejorar la forma en c\'omo las usamos o la forma que podemos realizar
tomas de decisiones a la hora de solucionar problemas computacionales o bien adquirir un equipo.
Es claro que en un mundo computacional, considero, es importante conocer en forma general y estar
pendientes de los avances tecnol\'ogicos dado que estos son constantes y son de gran relevancia
para cada persona, casi que sin importancia de \'ambito laboral pues la penetraci\'on de la
computaci\'on en la sociedad moderna es muy grande. Por ello esta investigacion se vuelve relevante.

\begin{thebibliography}{00}
\bibitem{Sta19} W. Stallings. \textit{Computer organization and architecture}. $11^{o}$ edici\'on. Hoboken: Pearson Education, 2019.

\end{thebibliography}
\end{document}
